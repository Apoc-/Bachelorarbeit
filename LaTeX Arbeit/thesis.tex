\documentclass[12pt,oneside,a4paper,parskip]{scrbook}
\usepackage[utf8]{inputenc}
\usepackage{csquotes}
\usepackage[ngerman]{babel}
\usepackage{floatflt}
\usepackage{subfigure}
\usepackage[pdftex]{graphicx}
\usepackage[hidelinks]{hyperref}
\usepackage{color}
\usepackage{amssymb}
\usepackage{textcomp}
\usepackage{nicefrac}
\usepackage{scrhack}
\usepackage{pdfpages}
\usepackage{float}
\usepackage{pdflscape}
\usepackage{subfigure}
\usepackage{pdfpages}
\usepackage[verbose]{placeins}
\usepackage[nouppercase,headsepline,plainfootsepline]{scrpage2}
\usepackage{listings}
\usepackage{xcolor}
\usepackage{color}
\usepackage{caption}
\usepackage{subfigure}
\usepackage{epstopdf}
\usepackage{longtable}
\usepackage{setspace}
\usepackage{booktabs}
\usepackage{hyperref}
\usepackage[style=numeric,backend=bibtex]{biblatex}
\bibliography{literatur}


%%%%%%%%%%%%%%%%%%%
%% definitions
%%%%%%%%%%%%%%%%%%%
\def\BaAuthor{René Ziegler}
\def\BaTitle{Über die Automatisierung der Entwicklung von Software Generatoren}
\def\BaSupervisorOne{Prof.\ Dr.\ Peter Braun}
\def\BaSupervisorTwo{M.Sc. Tobias Fertig}
\def\BaDeadline{\today}

\hypersetup{
pdfauthor={\BaAuthor},
pdftitle={\BaTitle},
pdfsubject={Subject},
pdfkeywords={Keywords}
}

%%%%%%%%%%%%%%%%%%%
%% configs to include
%%%%%%%%%%%%%%%%%%%
\colorlet{punct}{red!60!black}
\definecolor{background}{HTML}{EEEEEE}
\definecolor{delim}{RGB}{20,105,176}
\colorlet{numb}{magenta!60!black}

\definecolor{gray}{rgb}{0.4,0.4,0.4}
\definecolor{darkblue}{rgb}{0.0,0.0,0.6}
\definecolor{cyan}{rgb}{0.0,0.6,0.6}

\definecolor{pblue}{rgb}{0.13,0.13,1}
\definecolor{pgreen}{rgb}{0,0.5,0}
\definecolor{pred}{rgb}{0.9,0,0}
\definecolor{pgrey}{rgb}{0.46,0.45,0.48}

\lstset{
  basicstyle=\ttfamily,
  columns=fullflexible,
  showstringspaces=false,
  commentstyle=\color{gray}\upshape
  linewidth=\textwidth
}

\lstdefinelanguage{json}{
    basicstyle=\normalfont\ttfamily,
    numbers=left,
    numberstyle=\scriptsize,
    stepnumber=1,
    numbersep=8pt,
    showstringspaces=false,
    breaklines=true,
    backgroundcolor=\color{background},
    literate=
     *{0}{{{\color{numb}0}}}{1}
      {1}{{{\color{numb}1}}}{1}
      {2}{{{\color{numb}2}}}{1}
      {3}{{{\color{numb}3}}}{1}
      {4}{{{\color{numb}4}}}{1}
      {5}{{{\color{numb}5}}}{1}
      {6}{{{\color{numb}6}}}{1}
      {7}{{{\color{numb}7}}}{1}
      {8}{{{\color{numb}8}}}{1}
      {9}{{{\color{numb}9}}}{1}
      {:}{{{\color{punct}{:}}}}{1}
      {,}{{{\color{punct}{,}}}}{1}
      {\{}{{{\color{delim}{\{}}}}{1}
      {\}}{{{\color{delim}{\}}}}}{1}
      {[}{{{\color{delim}{[}}}}{1}
      {]}{{{\color{delim}{]}}}}{1},
}

\lstset{language=xml,
  morestring=[b]",
  morestring=[s]{>}{<},
  morecomment=[s]{<?}{?>},
  stringstyle=\color{black},
  numbers=left,
  numberstyle=\scriptsize,
  stepnumber=1,
  numbersep=8pt,
  identifierstyle=\color{darkblue},
  keywordstyle=\color{cyan},
  backgroundcolor=\color{background},
  morekeywords={xmlns,version,type}% list your attributes here
}

\lstset{language=Java,
  showspaces=false,
  showtabs=false,
  tabsize=4,
  breaklines=true,
  keepspaces=true,
  numbers=left,
  numberstyle=\scriptsize,
  stepnumber=1,
  numbersep=8pt,
  showstringspaces=false,
  breakatwhitespace=true,
  commentstyle=\color{pgreen},
  keywordstyle=\color{pblue},
  stringstyle=\color{pred},
  basicstyle=\ttfamily,
  backgroundcolor=\color{background},
%  moredelim=[il][\textcolor{pgrey}]{$$},
%  moredelim=[is][\textcolor{pgrey}]{\%\%}{\%\%}
}




\begin{document}


%%%%%%%%%%%%%%%%%%%
%% Titelseite
%%%%%%%%%%%%%%%%%%%


\frontmatter
\titlehead{%  {\centering Seitenkopf}
  {Hochschule für angewandte Wissenschaften Würzburg-Schweinfurt\\
   Fakultät Informatik und Wirtschaftsinformatik}}
\subject{Bachelorarbeit}
\title{\BaTitle\\[15mm]}
\subtitle{\normalsize{vorgelegt an der Hochschule f\"{u}r angewandte Wissenschaften W\"{u}rzburg-Schweinfurt in der Fakult\"{a}t Informatik und Wirtschaftsinformatik zum Abschluss eines Studiums im Studiengang Informatik}}
\author{\BaAuthor}
\date{\normalsize{Eingereicht am: \BaDeadline}}
\publishers{
  \normalsize{Erstpr\"{u}fer: \BaSupervisorOne}\\
  \normalsize{Zweitpr\"{u}fer: \BaSupervisorTwo}\\
}

%\uppertitleback{ }
%\lowertitleback{ }

\maketitle


%%%%%%%%%%%%%%%%%%%
%% abstract
%%%%%%%%%%%%%%%%%%%

\section*{Zusammenfassung}

TODO

\section*{Abstract}

TODO

\newpage
\chapter*{Danksagung}



%%%%%%%%%%%%%%%%%%%
%% Inhaltsverzeichnis
%%%%%%%%%%%%%%%%%%%
\tableofcontents



%%%%%%%%%%%%%%%%%%%
%% Main part of the thesis
%%%%%%%%%%%%%%%%%%%
\mainmatter

\chapter{Einführung}\label{ch:intro}

Als Henry Ford 1913 die Produktion des Model T, umgangssprachlich auch Thin Lizzy genannt, auf Fließbandfertigung umstellte revolutionierte er die Automobilindustrie. Ford war nicht der erste der diese Form der Automatisierung verwendete, bereits 1830 kam in den Schlachthöfen von Chicago eine Maschine zum Einsatz die an Fleischerhaken aufgehängte Tierkörper durch die Schlachterei transportierte. Die gefeierte Neuerung die Herstellung eines Automobils in Arbeitsschritte einzuteilen und diese jeweils an Arbeitsstationen durchführen zu lassen, wurde ebenfalls zuvor bereits von Ransom Eli Olds zur Produktion des Oldsmobile Curved Dash eingesetzt. Die Revolution war die Kombination aus beidem. In einer Produktionsstraße wurden die Karossen automatisch getaktet auf einem Fließband von Arbeitsstation zu Arbeitsstation befördert. An jeder Haltestelle wurden nur wenige Handgriffe von spezialisierten Arbeitern durchgeführt \cite{sagerso2008modelt}.

Fords Idee ein Auto für jedermann zu schaffen wurde durch diese Techniken Realität. Die Produktionszeit der Thin Lizzy wurde von 12,5 Stunden auf etwa 6 Stunden reduziert. War das Model T zu Beginn nur reichen Menschen vorbehalten, fiel der ursprüngliche Preis von 825\$ auf den tiefsten Stand von 259\$ \cite{reichlesz2010modelt}. Laut einer Informationsseite der US-Botschaft in Deutschland war das Durchschnittseinkommen in den USA 1910 bei jährlich 438\$, ein Fahrrad kostete 11,95\$ \cite{usembassyodnumbers}.

Im Zuge der weiter Entwicklung der Robotik wurden immer mehr Aufgaben die bisher von Menschen am Fließband durchgeführt wurden von Automaten durchgeführt. In der Automobil Industrie war General Motors der erste Hersteller bei welchem die Produktionsstraßen im Jahr 1961 mit 66 Robotern des Typs Unimation ausgestattet wurden. Bis zur Erfindung des integrierten Schaltkreises in den 1970ern waren diese ineffizient, jedoch explodierte der Markt für industrielle Roboter in den Folgejahren. Im Jahr 1984 waren weltweit ungefähr 100.000 Roboter im Einsatz \cite{wallen2008robohistory} \cite{czaeis2000genprog}.

Die Autoren Czarnecki und Eisenecker der Monographie Generative Programming sehen in dieser Industriellen Revolution einige Parallelen zur automatischen Softwareentwicklung. Eine Effizienzsteigerung könne erreicht werden wenn man zum einen die einzelnen Komponenten einer Softwarefamilie derart gestaltet, dass diese austauschbar in eine gemeinsame Struktur integriert werden können. Des weiteren müsse klar definiert sein welche Teile eines Programms konfigurierbar seien und welche der einzelnen Komponenten in welcher Konfiguration benötigt würden. Setzt man dieses definierte Wissen in Programmcode um, so dass dieser eine Software in der Entsprechenden Konfiguration generieren kann, so sei dies vergleichbar zur erfolgreichen Entwicklung der Automobilindustrie von der Manufaktur zur automatisierten Fabrik. Angefangen 1901 mit der Erfindung von auswechselbaren Teilen bei Ransom Olds, über die Weiterentwicklung dieses Konzept unter Verwendung von Fließbändern bei Ford im Jahr 1913, bis hin zur abschließenden Automatisierung mit Industriellen Robotern in den frühen 1980gern \cite{czaeis2000genprog}.

\section{Motivation}

Die Automation Assumption besagt:\\
\emph{\glqq If you can compose components manually, you can also autmate this process.\grqq} \cite{czaeis2000genprog}

Geht man von dieser Annahme aus, so sind Softwaregeneratoren ebenfalls automatisch generierbar. Hierfür müssen diese hinreichend in einer Komponenten basierten Struktur definiert sein und eindeutige Beschreibungen über die konfigurierbaren Anteile des Generators existieren.



\section{Zielsetzung}
\section{Aufbau der Arbeit}






\chapter{Grundlagen}

\chapter{Problemstellung}

\chapter{Lösung}

\begin{lstlisting}[label=lst:java,
				   language=java,
				   firstnumber=1,
				   caption=Beispiel für einen Quelltext]

public void foo() {
	// Kommentar
}
\end{lstlisting}

Lorem ipsum dolor sit amet, consectetur adipiscing elit. Ut vehicula felis lectus, nec aliquet arcu aliquam vitae. Quisque laoreet consequat ante, eget pretium quam hendrerit at. Pellentesque nec purus eget erat mattis varius. Nullam ut vulputate velit. Suspendisse in dui in eros iaculis tempus. Phasellus vel est arcu. Vestibulum ante ipsum primis in faucibus orci luctus et ultrices posuere cubilia Curae; Integer elementum, nulla eu faucibus dignissim, orci justo imperdiet lorem, luctus consectetur orci orci a nunc.

Praesent at nunc nec tortor viverra viverra. Morbi in feugiat lectus. Vestibulum iaculis ipsum at eros viverra volutpat in id ipsum. Donec condimentum, ligula viverra pharetra tincidunt, nunc dui malesuada nisi, vitae mollis lacus massa quis velit. Integer feugiat ipsum a volutpat scelerisque. Nulla facilisis augue nunc. Curabitur eget consectetur nulla. Integer accumsan sem non nisi tristique dictum.

Sed lacinia eu dolor sed congue. Ut dui orci, venenatis id interdum rhoncus, mattis elementum massa. Proin venenatis elementum purus ut rutrum. Phasellus sit amet enim porta, commodo mauris a, bibendum tortor. Nulla ut lobortis justo. Aenean auctor mi nec velit fermentum, quis ultricies odio viverra. Maecenas ultrices urna vel erat ornare, quis suscipit odio molestie. Donec vel dapibus orci, vel tincidunt orci.

Etiam vitae eros erat. Praesent nec accumsan turpis, et mollis eros. Praesent lacinia nulla at neque porta aliquam. Quisque elementum neque ac porta suscipit. Nulla volutpat luctus venenatis. Aliquam imperdiet suscipit pretium. Nunc feugiat lacinia aliquet. Mauris ut sapien nec risus porttitor bibendum. Aenean feugiat bibendum lectus, id mattis elit adipiscing at. Pellentesque interdum felis non risus iaculis euismod fermentum nec urna. Nullam lacinia suscipit erat ac ullamcorper. Sed vitae nulla posuere, posuere sem id, ultricies urna. Maecenas eros lorem, tempus non nulla vitae, ullamcorper egestas nibh. Vestibulum facilisis ante vel purus accumsan mattis. Donec molestie tempor eros, a gravida odio congue posuere.

Sed in tempus elit, sit amet suscipit quam. Ut suscipit dictum molestie. Etiam quis porta mauris. Cras dapibus sapien eget sem porta, ut congue sapien accumsan. Maecenas hendrerit lobortis mauris ut hendrerit. Suspendisse at aliquet est. Quisque eros est, scelerisque ac orci quis, placerat suscipit lorem. Phasellus rutrum enim non odio ullamcorper, sit amet auctor nulla fringilla. Nunc eleifend vulputate dui, a sollicitudin tellus venenatis non. Cras condimentum lorem at ultricies vestibulum. Vestibulum interdum lobortis commodo. Nullam rhoncus interdum massa, ut varius nisi scelerisque id. Nunc interdum quam in enim bibendum vulputate.


\chapter{Evaluierung}

Lorem ipsum dolor sit amet, consectetur adipiscing elit. Ut vehicula felis lectus, nec aliquet arcu aliquam vitae. Quisque laoreet consequat ante, eget pretium quam hendrerit at. Pellentesque nec purus eget erat mattis varius. Nullam ut vulputate velit. Suspendisse in dui in eros iaculis tempus. Phasellus vel est arcu. Vestibulum ante ipsum primis in faucibus orci luctus et ultrices posuere cubilia Curae; Integer elementum, nulla eu faucibus dignissim, orci justo imperdiet lorem, luctus consectetur orci orci a nunc.

Praesent at nunc nec tortor viverra viverra. Morbi in feugiat lectus. Vestibulum iaculis ipsum at eros viverra volutpat in id ipsum. Donec condimentum, ligula viverra pharetra tincidunt, nunc dui malesuada nisi, vitae mollis lacus massa quis velit. Integer feugiat ipsum a volutpat scelerisque. Nulla facilisis augue nunc. Curabitur eget consectetur nulla. Integer accumsan sem non nisi tristique dictum.

Sed lacinia eu dolor sed congue. Ut dui orci, venenatis id interdum rhoncus, mattis elementum massa. Proin venenatis elementum purus ut rutrum. Phasellus sit amet enim porta, commodo mauris a, bibendum tortor. Nulla ut lobortis justo. Aenean auctor mi nec velit fermentum, quis ultricies odio viverra. Maecenas ultrices urna vel erat ornare, quis suscipit odio molestie. Donec vel dapibus orci, vel tincidunt orci.

Etiam vitae eros erat. Praesent nec accumsan turpis, et mollis eros. Praesent lacinia nulla at neque porta aliquam. Quisque elementum neque ac porta suscipit. Nulla volutpat luctus venenatis. Aliquam imperdiet suscipit pretium. Nunc feugiat lacinia aliquet. Mauris ut sapien nec risus porttitor bibendum. Aenean feugiat bibendum lectus, id mattis elit adipiscing at. Pellentesque interdum felis non risus iaculis euismod fermentum nec urna. Nullam lacinia suscipit erat ac ullamcorper. Sed vitae nulla posuere, posuere sem id, ultricies urna. Maecenas eros lorem, tempus non nulla vitae, ullamcorper egestas nibh. Vestibulum facilisis ante vel purus accumsan mattis. Donec molestie tempor eros, a gravida odio congue posuere.

Sed in tempus elit, sit amet suscipit quam. Ut suscipit dictum molestie. Etiam quis porta mauris. Cras dapibus sapien eget sem porta, ut congue sapien accumsan. Maecenas hendrerit lobortis mauris ut hendrerit. Suspendisse at aliquet est. Quisque eros est, scelerisque ac orci quis, placerat suscipit lorem. Phasellus rutrum enim non odio ullamcorper, sit amet auctor nulla fringilla. Nunc eleifend vulputate dui, a sollicitudin tellus venenatis non. Cras condimentum lorem at ultricies vestibulum. Vestibulum interdum lobortis commodo. Nullam rhoncus interdum massa, ut varius nisi scelerisque id. Nunc interdum quam in enim bibendum vulputate.

\chapter{Zusammenfassung}


\backmatter
%%%%%%%%%%%%%%%%%%%
%% create figure list
%%%%%%%%%%%%%%%%%%%

\listoffigures
\addcontentsline{toc}{chapter}{Verzeichnisse}

%%%%%%%%%%%%%%%%%%%
%% create tables list
%%%%%%%%%%%%%%%%%%%
\listoftables

%%%%%%%%%%%%%%%%%%%
%% create listings list
%%%%%%%%%%%%%%%%%%%
%\lstlistoflistings
%\addcontentsline{toc}{chapter}{Listings}

\printbibliography
\addcontentsline{toc}{chapter}{Literatur}

%%%%%%%%%%%%%%%%%%%
%% declaration on oath
%%%%%%%%%%%%%%%%%%%

\addchap{Eidesstattliche Erklärung}

Hiermit versichere ich, dass ich die vorgelegte Bachelorarbeit selbstständig verfasst und noch nicht anderweitig zu Prüfungszwecken vorgelegt habe. Alle benutzten Quellen und Hilfsmittel sind angegeben, wörtliche und sinngemäße Zitate wurden als solche gekennzeichnet.

\vspace{20pt}
\begin{flushright}
$\overline{~~~~~~~~~~~~~~~~~\mbox{\BaAuthor, am \today}~~~~~~~~~~~~~~~~~}$
\end{flushright}

\addchap{Zustimmung zur Plagiatsüberprüfung}

Hiermit willige ich ein, dass zum Zwecke der Überprüfung auf Plagiate meine vorgelegte Arbeit in digitaler Form an PlagScan (www.plagscan.com) übermittelt und diese vorrübergehend (max. 5~Jahre) in der von PlagScan geführten Datenbank gespeichert wird sowie persönliche Daten, die Teil dieser Arbeit sind, dort hinterlegt werden.

\begin{small}
Die Einwilligung ist freiwillig. Ohne diese Einwilligung kann unter Entfernung aller persönlichen Angaben und Wahrung der urheberrechtlichen Vorgaben die Plagiatsüberprüfung nicht verhindert werden. Die Einwilligung zur Speicherung und Verwendung der persönlichen Daten kann jederzeit durch Erklärung gegenüber der Fakultät widerrufen werden.
\end{small}

\vspace{20pt}
\begin{flushright}
$\overline{~~~~~~~~~~~~~~~~~\mbox{\BaAuthor, am \today}~~~~~~~~~~~~~~~~~}$
\end{flushright}

\end{document}
